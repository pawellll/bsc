\section{Wprowadzenie}
\subsection{Przedmiot pracy}
% music emotion recognition: sttate of the art review
Muzyka towarzyszyła człowiekowi od czasów prehistorycznych. Z~czasem stała się jedną z~form sztuki. Niewątpliwie, gdy słyszymy jakaś melodię, nie sprawia nam większego kłopotu, aby określić emocje z nią związane. Należy jednak mieć na uwadze źródło emocji. Rozróżnić możemy emocje wyrażane przez muzykę oraz przez nią indukowane. Tematem niniejszej pracy jest rozpoznawanie nastroju muzyki przy użyciu uczenia maszynowego. Przez określenie ,,nastrój muzyki'' mamy tutaj na myśli emocje, które ta muzyka reprezentuje. 
% W dalszej części pracy określenia "nastrój" odraz "emocje" będą używane naprzemiennie. 
Konkretnym narzędziem wybranym w celu klasyfikacji muzyki jest sztuczna sieć neuronowa. Idea klasyfikacji utworów muzycznych pod tym kątem jest względnie nowa, lecz można zauważyć wzrastające zainteresowanie tym tematem\cite{stateOfArt}. Do tej pory powstał szereg różnych prac podejmujących to zadanie\cite{musicANN1}\cite{musicANN2}\cite{musicANN3}. Niektóre korzystają nie tylko z samego sygnału audio, ale także np. z tekstu utworu\cite{musicLyrics}. W tej pracy jednak pod uwagę brany jest jedynie sygnał audio z~którego wyekstrahowano odpowiednie cechy, które mogłyby pozwolić sieci neuronowej rozpoznawać emocje reprezentowane przez dany utwór muzyczny. 
%Celem pracy jest stworzenie sztucznej sieci neuronowej, która będzie z powodzeniem potrafiła rozpoznawać nastrój %utworu bazując na wyekstrahowanych cecha utworu.

W kolejnych rozdziałach zostały opisane niezbędne podstawy teoretyczne, których przyswojenie pozwala zrozumieć w~jaki sposób postawione zadanie jest realizowane. W rozdziale \ref{rozdzial_sieci} znajduje się krótki opis sieci neuronowych, w rozdziale \ref{rozdzial_ekstrakcja} czytelnik dowie się o tym jakie konkretnie atrybuty muzyki były rozważane, natomiast w rozdziale \ref{rozdzial_modelEmocji} opisany jest sposób w~jaki matematyka pomaga nam w~opisaniu emocji. Opis stworzonego systemu można znaleźć w rozdziale \ref{rozdzial_system}. Wyniki oraz wnioski wynikające z podjętej próby podejścia do opisywanego zagadnienia zostały przedstawione w rozdziale kolejno \ref{rozdzial_wyniki} oraz \ref{rozdzial_wnioski}.
\subsection{Problematyka pracy}
Analiza muzyki pod kątem emocji jest zadaniem, które nie wiąże się tylko z przetwarzaniem sygnałów oraz uczeniem maszynowym, ale także z psychologią muzyki oraz jej teorią. Jest to bardzo wymagający problem, gdyż nastrój utworu muzycznego może być wysoce subiektywnym odczuciem. Wpływ na ocenę mogą mieć także wspomnienia danej osoby, nastrój w~danej chwili, indywidualne preferencje czy poziom wykształcenia muzycznego. Wszystkie wspomniane jednak kwestie nie są na tyle znaczące, aby podjęcie pracy nad tym tematem było niemożliwe czy całkowicie nieskuteczne. Oczywiście nie da stworzyć się systemu, który będzie działał niezawodnie, bo tak jak zostało to wspomniane, zbyt wiele indywidualnych czynników ma wpływ na percepcje człowieka. Dotychczasowe badania udowadniają, że jest możliwe sprostanie temu zadaniu w zadowalającym stopniu\cite{stateOfArt} i~taka próba zostaje podjęta w~niniejszej pracy.